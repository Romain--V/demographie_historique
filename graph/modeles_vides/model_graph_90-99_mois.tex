\documentclass[a4paper,landscape]{article}

\usepackage[francais]{babel}
\usepackage[utf8]{inputenc}
\usepackage[T1]{fontenc}
\usepackage{tikz}
\usepackage{pgfplots}

\usepackage[top=1cm, bottom=1cm, left=1cm, right=1cm]{geometry}


\begin{document}



\begin{tikzpicture}
\begin{axis}[
 % ybar, % remplace la ligne (par defaut) par les barre d'histogramme
  %axis x line=bottom,  % affiche uniquement axe bas
  %axis y line=left, % affiche axe gauche
  %width=25cm, 
 % height=15cm,
  xlabel={Dates}, %label x
  ylabel={Nombre},  % label y
 % legend entries={Conceptions,Sépultures,Mariages},
  title={La famine}, % titre du graphique
  xtick={1,2,3,4,5,6,7,8,9,10,11,12,13,14,15,16,17,18,19,20,21,22,23,24,25,26,27,28,29,30,31,32,33,34,35,36,37,38,39,40,41,42,43,44,45,46,47,48,49,50,51,52,53,54,55,56,57,58,59,60,61,62,63,64,65,66,67,68,69,70,71,72,73,74,75,76,77,78,79,80,81,82,83,84,85,86,87,88,89,90,91,92,93,94,95,96,97,98,99,100,101,102,103,104,105,106,107,108,109,110,111,112,113,114,115,116,117,118,119,120,121,122,123,124,125,126,127,128,129}, %rappel des valeurs x pour eviter les pb de placement
  xticklabels={1689-04,1689-05,1689-06,1689-07,1689-08,1689-09,1689-10,1689-11,1689-12,1690-01,1690-02,1690-03,1690-04,1690-05,1690-06,1690-07,1690-08,1690-09,1690-10,1690-11,1690-12,1691-01,1691-02,1691-03,1691-04,1691-05,1691-06,1691-07,1691-08,1691-09,1691-10,1691-11,1691-12,1692-01,1692-02,1692-03,1692-04,1692-05,1692-06,1692-07,1692-08,1692-09,1692-10,1692-11,1692-12,1693-01,1693-02,1693-03,1693-04,1693-05,1693-06,1693-07,1693-08,1693-09,1693-10,1693-11,1693-12,1694-01,1694-02,1694-03,1694-04,1694-05,1694-06,1694-07,1694-08,1694-09,1694-10,1694-11,1694-12,1695-01,1695-02,1695-03,1695-04,1695-05,1695-06,1695-07,1695-08,1695-09,1695-10,1695-11,1695-12,1696-01,1696-02,1696-03,1696-04,1696-05,1696-06,1696-07,1696-08,1696-09,1696-10,1696-11,1696-12,1697-01,1697-02,1697-03,1697-04,1697-05,1697-06,1697-07,1697-08,1697-09,1697-10,1697-11,1697-12,1698-01,1698-02,1698-03,1698-04,1698-05,1698-06,1698-07,1698-08,1698-09,1698-10,1698-11,1698-12,1699-01,1699-02,1699-03,1699-04,1699-05,1699-06,1699-07,1699-08,1699-09,1699-10,1699-11,1699-12} %label des point absyss.
  ]
  
 % \addplot+[mark=none,smooth] coordinates {(1,30)  };

 

\end{axis}
\end{tikzpicture}



\end{document}
